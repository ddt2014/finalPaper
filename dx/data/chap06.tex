\chapter{总结与展望}

\section{工作总结}

随着软、硬件技术的不断提升,摄像头的各项性能指标的逐步提高,但造价却越来越便宜。这就使得对于摄像头的大规模应用成为可能,关于包含多个相同或不同摄像头的多摄像头系统的研究也日趋增多,多摄像头系统的应用场景也越来越广泛。而对于多摄像头系统来说,由于系统内可能存在的各种误差、延时,如何将所有摄像头的拍摄时间进行同步,成为了一个系统研究的关键问题,而本文则提出了一个基于LED点阵检测系统的多摄像头系统的时间同步方法。

该方法利用FPGA与LED点阵组成检测系统,使得LED点阵不断变化各个LED灯的亮灭组合方式,形成一系列的LED点阵状态。当摄像头拍摄到LED点阵时,根据拍摄图像中的LED点阵状态,可以判断该状态在状态序列当中所处的位置,从而可以根据各个状态的持续时间检测出摄像头的拍摄时间。而根据LED点阵编码方法的不同,这个状态序列可以呈现出不同的显示规律,摄像头在进行拍摄时也能够得到不同的拍摄效果,导致不同的检测精度。通过对不同编码方法的实验对比,可以发现进位流水编码方法的状态序列长,检测精度高,而且具有叠加可识别性,同时可以适用于大多数摄像头,对摄像头没有过多的性能、参数要求。

在获得摄像头的拍摄时间后,可以根据该时间计算多摄像头系统内各个摄像头之间的时间差,即当前系统的同步误差。然后选取拍摄时间的中间值,控制各个摄像头进行拍摄时间的调整。对摄像头拍摄时间的调整主要有拍摄暂停和帧率变换两种方法,这两种方法都能够取得较好的调整效果,使得系统内所有摄像头最终实现拍摄时间的同步。

在实际应用中,本文利用一台图像处理服务器和四台树莓派电脑搭建了一个多摄像头系统。并利用该系统对上述各方法进行验证。在克服了卷帘快门摄像头果冻效应和多摄像头视野校准等问题之后,使用该系统对LED点阵的各种编码方法进行了比较,同时也对摄像头拍摄时间的检测方法进行了验证,还对该系统进行了时间同步实验,都取得了很好的实验结果。

\section{未来展望}

在后续工作中,为了提高检测精度,可以扩大LED点阵中包含的LED灯的数量。目前的检测精度受果冻效应的影响,当点阵中每行LED灯的数量增多时,可以使得每个LED灯的亮灭变化加快,从而提高检测精度。同时在进行系统同步的过程中,受系统内随机延时的影响需要多次迭代逐渐同步。而在后期系统优化过程中可以通过大量重复性试验,从随机延时中挖掘出一定规律,从而提高系统同步速度。

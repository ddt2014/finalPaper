\chapter{数据库与测量分析}
\label{cha:database and measurement}

\section{引言}
由于本文主要研究的是演员在社交媒体上对电视剧的推广作用,涉及到电视剧、演员及微博信息。本章首先对电视剧及演员相关数据的数据库进行说明,然后对研究过程中需要的演员、电视剧、微博的特征及其提取过程进行介绍。之后对数据进行测量和分析,包括对微博影响力、话题热度、推广模式和他们之间的关系的研究和分析。

\section{数据集}
\subsection{数据库}
数据库系统包括演员数据库和电视剧数据库两部分,如图3.1。数据获取来源是爱奇艺\footnote{http://www.iqiyi.com/}、微博\footnote{http://www.weibo.com/}和豆瓣\footnote{http://www.douban.com/}。通过爬虫模拟登陆、模拟搜索的形式获取。

\begin{figure}[H] 
  \centering
  \includegraphics[height=4.5cm,width=8cm]{4}
  \caption{Data of the database}
  \label{4}
\end{figure}

爱奇艺是国内领先的视频网站,提供海量、优质的电视剧、电影、网络视频服务,网罗了中国最广大的年轻用户群体。爱奇艺打造涵盖电影、电视剧、综艺、动漫在内的十余种类型的中国最大正版视频内容库,也是中国付费用户规模最大的视频网站 \cite{http://www.iqiyi.com/common/aboutus.html}。爱奇艺数据爬虫获取爱奇艺电视剧页面每天的电视剧列表,然后访问每个电视剧的详情页,如图3.2,获得电视剧的相关信息,包括电视剧名称、主演、导演、类型、集数、播放量、简介等信息。

\begin{figure}[H] 
  \centering
  \includegraphics[height=4.5cm,width=8cm]{爱奇艺}
  \caption{电视剧欢乐颂详情页}
  \label{爱奇艺}
\end{figure}

豆瓣是一个社区网站,提供关于书籍、电影、音乐等作品的信息,无论描述还是评论都由用户提供(User-generated content, UGC),是中国Web 2.0网站中具有特色的一个网站\cite{https://zh.wikipedia.org/wiki/豆瓣}。对于电影和电视剧的豆瓣评分能反应当前广大群众对其质量、水平的认同情况。豆瓣爬虫根据爱奇艺爬虫获得的电视剧列表,从豆瓣搜索入口模拟搜索,获得该电视剧的豆瓣评分,以反应该电视剧的质量,如图3.3。

\begin{figure}[H] 
  \centering
  \includegraphics[height=4.5cm,width=8cm]{欢乐颂豆瓣}
  \caption{欢乐颂豆瓣主页}
  \label{欢乐颂豆瓣}
\end{figure}

微博是由新浪公司推出的提供微博客的服务网站。用户可以在微博网页、客户端发布140汉字(280字符)以内的信息,并可上传图片和链接视频,实现即时分享,同时,可以提供提供评论、转发、点赞功能。新浪微博是一个基于用户关系的信息分享、传播以及获取信息的平台,它占据中国微博用户总量的57\%,以及中国微博活动总量的87\%,是中国大陆访问量最大的网站之一 \cite{https://zh.wikipedia.org/wiki/新浪微博}。微博爬虫包括电视剧官方微博爬虫、电视剧话题爬虫和演员信息爬虫。

大部分电视剧在微博上都有官方微博,如图3.4,用来发布电视剧、演员相关的信息,比如电视剧预告、演员拍摄花絮、电视剧相关或者演员相关的推广信息等等,使得观众获得更多的关于电视剧和主演的信息和动态。电视剧官方微博爬虫根据爱奇艺爬虫获得的电视剧名称,获取电视剧在微博上官方微博的id,然后模拟访问主页,获得该电视剧官方微博的基本信息,包括id、粉丝数等,和发布的所有微博信息。

同时,大部分电视剧在微博上会有一到多个话题,以“\#话题名\#”的形式存在,如图3.5中欢乐颂话题主页,当用户在微博中发布微博讨论电视剧时会带上相关话题,使话题活跃。电视剧话题爬虫根据从电视剧官方微博中出现及微博搜索入口搜索电视剧名字,可以得到电视剧的话题名称,进而获得该电视剧话题的相关信息,包括话题阅读量、话题讨论量、话题粉丝量等。


\begin{figure}[H] 
  \centering
  \includegraphics[height=4.5cm,width=8cm]{欢乐颂官微}
  \caption{电视剧欢乐颂官方微博}
  \label{欢乐颂官微}
\end{figure}
\begin{figure}[H] 
  \centering
  \includegraphics[height=4.5cm,width=8cm]{欢乐颂话题}
  \caption{电视剧欢乐颂话题主页}
  \label{欢乐颂话题}
\end{figure}

爱奇艺爬虫获取电视剧的主演后可以建成一个演员库,演员信息爬虫根据演员名字在微博上模拟搜索和百度搜索的形式获得演员微博上对应的id,然后根据id获得演员的基本信息和发布的所有微博信息,如图3.6。基本信息包括演员id、昵称、性别、描述、注册时间、粉丝数、关注数、微博数等等。微博信息包括发布的微博内容、时间及其获得的点赞数、转发数、评论数。

\begin{figure}[H] 
  \centering
  \includegraphics[height=4.5cm,width=8cm]{刘涛}
  \caption{刘涛微博主页}
  \label{刘涛}
\end{figure}

我们的爬虫运行在服务器后台,爱奇艺爬虫每天对电视剧信息进行爬取,微博爬虫每隔一段时间(3个月)从微博上对演员信息增量爬取。电视剧数据获取时间从2013年1月到2016年12月31日,微博爬虫全量获取演员信息和发布的微博信息。

\subsection{特征提取}
为研究演员所发微博的影响力,我们以微博作为研究对象,因此需要获得微博的基本特征。推广电视剧的微博有三方面的特征,一是来自发微博的演员的特征,包括演员粉丝数、性别和作品数。二是来自该微博推广的电视剧的特征,包括电视剧的豆瓣评分和演员阵容。三是来自微博自身的特征,包括该微博的发布时间、日期和内容。
发布微博的演员的特征中,粉丝数代表该演员在微博中粉丝量也就是影响力,当两个不同粉丝量的演员发布同一条微博时,粉丝量大的演员发布的微博会被更多的人看到,也就更容易起到更好的推广作用;男演员和女演员的行为本身有差异,其粉丝男女构成也有差异,相应的粉丝行为和对推广微博的行为也会有差异;演员的作品数代表该演员在影视界和群众视听中的活跃程度,演员出演的电视剧越多,越容易被观众记住也越有名。微博推广的电视剧的特征中,电视剧的豆瓣评分代表了看过的群众对该电视剧质量评价;电视剧的演员阵容用该电视剧所有主演的粉丝数的总和来表示,代表了这个电视剧整体的粉丝量和可能的受关注度,越多名人参与到同一部电视剧中,这个电视剧越容易火。微博自身的特征是由发布微博的演员决定的,包括时间、日期和内容,这是我们研究的主要内容。

\begin{figure}[!htbp]
\centering
\includegraphics[height=5cm,width=8cm]{3}
\caption{Feature Extraction Module}
\label{3}
\end{figure}

对微博爬虫获取的微博,需要提取每一项微博的特征,因为微博自身的特征不需要从其他数据库或者爬虫获得,这里介绍前两方面特征的获取,如图3.7。对微博所推广电视剧的豆瓣评分从豆瓣爬虫爬取或者已爬取的数据库中搜索可得。该微博推广电视剧的微博的演员阵容,先从爱奇艺爬虫获取的电视剧信息中获得该电视剧的主演,然后从微博爬虫或者已爬取数据中获得这些演员的粉丝数目,求和得到阵容。发布微博的演员的作品数为统计爱奇艺爬取的所有电视剧中由该演员主演的数目。演员的性别由微博爬虫或已爬取的演员基本信息中获得。

\section{测量分析}
\subsection{微博影响力与话题热度}
(1) 微博影响力。
为了衡量一条微博的影响力,我们采用粉丝参与度作为评判标准,包括这条微博的转发数、评论数和点赞数,因为这些数据能反应看到并参与到电视剧推广中的用户数量。经过统计,所有演员的所有微博的转发数、评论数、点赞数的比例是${1: 1.86: 4.66}$,为了使这三项评判指标具有相同重要程度,因此赋予其权重比为${4.66: 2.51: 1}$,也能反应出转发、评论和点赞能带来的不同影响力。转发推广微博能使转发人的粉丝也看到,用户的参与感非常强,转发的人多了就具有滚雪球效应,能扩大宣传效果,因此其权重最大。用户通过评论推广微博参与到推广中,评论越多也越容易使该演员上微博热门,但评论并没有扩大用户群,所以用户参与感和影响力不如转发大。相比于转发和评论,用户点赞既不会扩大用户群也不需要输入文字,其用户参与感最弱,所以权重最小。定义微博影响力公式如下:
\begin{equation}P_i^j = 4.66 * W_p^i + 2.51 * W_v^i + W_a^i\end{equation}
其中,$P_i^j$ 表示与电视剧$j$相关的微博$i$的影响力,$W_p^i$, $W_v^i$和$W_a^i$分别表示微博$i$的粉丝转发数、评论数和点赞数。

那么对一部电视剧带来的微博影响力是它的所有主演的发布的推广微博的影响力之和:
\begin{equation}T_j = \sum_{i=1}^n P_i^j\end{equation}
其中$T_j$表示电视剧$j$所有主演的推广微博影响力之和,$n$为电视剧主演发布微博的数量。

(2) 电视剧微博话题热度。
我们研究的电视剧,每部在微博中都有对应的微博话题。每个微博话题都有相应的阅读量和讨论量,其中阅读量表示该话题的用户阅读数目,讨论量表示用户发布的带该话题名的微博数目。电视剧上映前和上映中进行宣传时,宣传方都会在发布微博时带话题发布,即带“\#话题名\#”的形式发布微博。话题讨论量和阅读量越多,表明越多的用户参与到该话题讨论中,越容易成为热门话题,进而达到更好的宣传效果。用户带话题名发布微博,会增加微博话题讨论量,使其粉丝看到该微博,提高微博话题阅读量。也就是说微博话题阅读量是话题讨论和宣传效果的体现,代表了该电视剧微博话题覆盖到的人数,是其热度的体现。因此我们选用电视剧的微博话题阅读量来衡量电视剧在微博中的话题热度:
\begin{equation}H_j = R_j\end{equation}
其中$H_j$表示电视剧$j$的微博话题热度,$R_j$表示微博话题的阅读量。

(3) 电视剧微博影响力与微博话题热度。
为检验电视剧微博影响力与话题热度的相关性,我们选用皮尔森相关系数(Pearson correlation coefficient)和最大信息系数(Maximal Information Coefficient, MIC)来检验。皮尔森相关系数可以反映出两个变量之间的线性相关程度,系数的变化范围为-1到1。系数值为1意味着变量X和Y呈完全正相关关系,Y随X增加而增加。系数值为-1意味着X和Y呈完全负相关关系,Y随X增大而减小。系数值为0意味着两个变量间没有线性关系。MIC是专门用于快速探索多维数据集的双变量依赖关系的度量。MIC是基于信息的非参数探索(MINE)统计方法系列的一部分,MINE不仅可用于识别数据集中的重要关系,还可用于表征数据集。MIC可以分析变量之间的广泛关系,而不限定于特定的函数类型\cite{10}。

\begin{table}[H]
\centering
\caption{Correlation between influence of microbloggings and topic hotness of TV series}
\begin{tabular}{|c|c|c|c|} \hline
variable x&varible y&PCC&MIC\\ \hline
influence&hotness&0.697&0.356\\
\hline\end{tabular}
\end{table}

经过统计,电视剧的微博影响力与最终话题热度高度相关,两者之间相关性如表1所示。因此,可以将微博影响力作为微博话题热度的评判标准,即在判断推广微博带来的营销效果时,推广电视剧的微博影响力大则说明该电视剧话题热度高。电视剧话题演化的和微博影响力的动态数据也验证了二者之间的相关性,以欢乐颂为例。电视剧微博话题热度与微博影响力每日变化如图,从图中可以看出,二者趋势和形状几乎一致。

\begin{figure}[!htbp]
\centering
\includegraphics[height=3.cm,width=8cm]{1}
\caption{Everyday data of influence of microblogs and topic hotness}
\end{figure}



\subsection{推广模式}
(1) 推广周期。
演员对电视剧的推广周期主要可分为三个阶段,一是电视剧的筹备、拍摄阶段,二是电视剧首播阶段即第一集播出前后,三是首播之后的阶段,以电视剧“杉杉来了”为例,如图3.8。演员在电视剧拍摄期间会分享一些电视剧拍摄地、拍摄进度、角色和剧情相关信息等内容,这些微博不但可以让粉丝对演员的动态有所了解,还可以让观众提前了解电视剧相关信息,引起粉丝的兴趣和期待,起到提前宣传的作用。在第一集播放的前后几天是演员和电视剧官方宣传的高峰期,演员会发布微博进行大力度推广,我们将电视剧首集首播日前后5天定义为电视剧的首播阶段。统计发现虽然有些演员在微博中不活跃,但是仍然会在首播前后,发布微博进行推广宣传。首播阶段是影响收视率的关键时间,在这个时期演员推广可以提醒和号召粉丝看这个电视剧,促进提高话题热度和收视率。首播之后,演员还会继续进行微博推广,发布微博传递信息或者与粉丝互动,能在维持现有观众继续观看后续剧集的同时,吸引更多观众参与话题讨论并提高电视剧收视率。除此之外,当电视剧登陆其他电视台开播时,演员也会发布微博进行推广。因此,从推广周期来看,演员推广微博时期分为三类, 分别为筹备阶段、首播阶段和首播后阶段。
\begin{figure}[!htp]
\centering
\includegraphics[width=7.8cm,height=4.5cm]{period}
\caption{Microbloggings distribution of different promotion period.}
\end{figure}

(2) 推广时间。
在一天中不同时间发布推广微博,由于用户使用习惯的差异,被用户看到的时长和概率不同,也就会达到不同的推广效果。比如同一条推广微博,在用户刷微博高峰和上班时间发布,肯定会获得不同的参与度。在一天中,午饭前、晚饭前及晚上通常都是用户刷微博的高峰时段。图3-8显示了演员发布微博时间的数目分布情况,同时也包括演员在电视剧上映前和上映后的数目差异。在本文中,我们想知道演员在不同时间段发布微博,哪个时间段会带来更好的推广效果,因此,我们将演员发布的微博根据发布时间分成三类,分别是上午(1时至12时),下午(12时至18时),晚上(18时至次日1时)。
\begin{figure}[!htbp]
\centering
\includegraphics[height=3.5cm,width=8cm]{time}
\caption{Distribution of actors' microblogs along the time}
\end{figure}

(3) 互动模式。
在微博上推广电视剧过程中,演员会与其他主演、粉丝、电视剧官方微博等有很多交互行为来增加用户的参与度,促进电视剧的推广。图3-9展示了电视剧欢乐颂推广过程中出现的互动关系。图中每个节点表示一个演员或者官方微博账号,节点大小代表了与其他账号互动的数目,发过的微博中与其他账号互动越多,节点大小越大。有向箭头表明了互动的方向,边的粗细代表互动的频率,频率越高边越粗。一条微博可以使演员间、演员与官方微博间、演员与粉丝间建立互动关系,如图3-9中为一条微博的传播情况,图中黄色点是该电视剧的主演,每个点是一个微博用户,每条边是一个转发关系。

\begin{figure}[!htbp]
\centering
\includegraphics[height=4cm,width=5.4cm]{微博互动}
\caption{Interaction relationships of one microblog for TV series "\textit{Huanlesong}"}
\end{figure}

\begin{figure}[!htbp]
\centering
\includegraphics[height=4cm,width=5.4cm]{interaction}
\caption{Interaction relationships of TV series "\textit{Huanlesong}"}
\end{figure}

演员发布推广微博时有四种互动模式:
a. 与电视剧其他主演互动。演员会转发其他主演微博,或者发布微博“@”其他主演,在微博中与其他主要进行交流。主演间互动一方面会增加 微博话题性,另一方面也会结合多方粉丝,扩大推广效应。
b. 与官方微博互动。经过统计发现,一种常见的推广模式是,官方微博作为源头,发布推广微博,并在微博中“@”主演,主演将会转发这条微 博,提升官微推广的影响力。我们用累计分布图来分析演员响应官方微博的时间概率分布。以电视剧"\textit{欢乐颂}"为例,78\%的主演的响应时间在2小时以下。统计发现,对所有电视剧而言见图3-9,80\%的主演的响应时间都会在2小时以下。
c. 原创非互动微博。此类微博由演员原创, 虽然不包含与其他人的互动,但通过表达针对电视剧剧情或者人物的感想和看法,更好的表达演员的 情感和想法,更能够吸引粉丝的注意力。
d. 其他互动模式。除了以上三种互动的其他模式称为其他互动模式,包括演员转发视频网站官方微博的微博、转发粉丝微博、转发宣传媒体微博等。

图3-9显示了所有演员微博中以上四种模式所占比例,其中也包括了在电视剧上映前后的比例差异。

\begin{figure}[h]
  \centering%
  \subcaptionbox{TV series: "\textit{Huanlesong}"}[4.5cm] 
    {\includegraphics[height=3.7cm,width=7.5cm]{53}}%
  \hspace{4em}%
  \subcaptionbox{All TV series}[4.5cm] 
      {\includegraphics[height=3.7cm,width=7.5cm]{52}}
  \caption{Response time of actors to official accounts' references}
  \label{fig:big1-subcaptionbox}
\end{figure}

\begin{figure}[!htbp]
\centering
\includegraphics[width=8.5cm,height=3.7cm]{6}
\caption{Proportion of interaction modes}
\end{figure}

(4) 推广模式与话题热度。
为验证各种推广模式与话题热度的相关性,我们用皮尔森相关系数和最大信息系数来检验。通过两种检测方法可以表明各个推广模式与话题阅读 量具有较强的相关性,因此接下来我们就可以检验各个推广模式对话题热度的因果性和有效程度。
\begin{table}[!htbp]
\centering
\caption{The correlation between promotion patterns and topic hotness}
\begin{tabular}{|c|c|c|c|} \hline
\multicolumn{2}{|c|}{promotion patterns}&PCC&MIC\\ \hline
\multirow{3}{*}{promotion period} & preparation&0.389&0.284\\% \hline
&premiere&0.515&0.342\\% \hline
&post-premiere&0.813&0.399\\ \hline
\multirow{3}{*}{promotion time} &morning&0.239&0.306\\% \hline
&afternoon&0.542&0.407\\% \hline
&evening&0.856&0.340\\ \hline
\multirow{3}{*}{interaction mode} &actors&0.390&0.292\\% \hline
&official accounts&0.745&0.356\\% \hline
&original&0.600&0.297\\ 
&other&0.630&0.349\\ 
\hline\end{tabular}
\end{table}

\label{sec:multifig}


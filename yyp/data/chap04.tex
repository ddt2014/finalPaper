\chapter{演员社交推广行为的影响力模型}
\label{model}

\section{引言}
理想情况下,检验策略有效性的黄金标准是基于实验的方法,比如做A/B测试,不同的策略可以随机分配给用户,但是完全随机对照实验存在很多限制,比如成本非常高,操作困难,在实际环境中不可行。而采用非随机对照实验则容易出现组间基线不齐,数据偏倚。因此提供一个统计方法而不是实验方法来检测策略的有效性是非常重要和有必要的。倾向值匹配算法解决了以上问题,它将混淆变量纳入Logistic模型, 归纳成一个倾向值,在进行倾向值匹配时,混淆变量得以控制,使得对最终结果的影响只能归因于自变量[11]。比如, 将微博发布时间作为自变量,控制其他混淆变量近似相等,如选择同一个演员,在电视剧首播期间, 发布内容相似的原创微博,因此最终这些微博影响力的差异只能归因于发布时间。

\section{倾向值匹配算法}
\subsection{混淆变量}
在倾向值匹配算法中,混淆变量是指,除了研究策略外的其他可能会影响策略有效性的变量。如演员A和演员B发布的微博带来的影响力的不同,除了策略的不同,还可能是因为二者的粉丝量不同带来的差异。这些混淆变量对推广效果的影响就是选择性误差,在倾向值匹配算法中,通过控制混淆变量来遏制选择性误差对结果的影响。本章中,选用微博的与演员相关的特征粉丝量作为混淆变量,记做$X$。为压缩粉丝数数据尺度,对粉丝数取10的对数。

\subsection{推广策略}
为研究演员社交行为对电视剧的推广作用,将演员发布的推广微博作为研究对象,自变量是三类推广模式下的10项推广策略,如表4-2。我们定义每一项推广策略都是一项策略t,其值为T。在我们的研究中,$T={0,1}$,其中“1”代表采用这项策略,“0”代表不采用这项策略。同时,我们定义与采用干预t的微博的影响力作为输入$Y(t)$。当研究一项推广策略时,另外的推广模式下的推广策略会作为混淆变量加入变量$X$。

\begin{table}[!htbp]
\centering
\caption{Promotion patterns and promotion strategies}
\begin{tabular}{|c|c|} \hline
Promotion Pattern& Strategy\\ \hline
\multirow{3}{*}{Promotion period} &preparation\\% \hline
&Premiere period\\% \hline
&Post-premiere period\\  
\hline
\multirow{3}{*}{Promotion time} &morning\\% \hline
&afternoon\\% \hline
&evening\\ \hline
\multirow{4}{*}{Interaction mode} &actors\\% \hline
&official accounts\\% \hline
&original microblog\\ 
&others\\ 
\hline\end{tabular}
\end{table}

\subsection{算法步骤}
如果两条微博唯一的差异是采用或没采用一项策略,那么我们可以合理将这两条微博最终影响力的差异完全归因于这项策略的有效程度。这也告诉我们,可以通过控制其他变量都相同或者类似来计算策略的有效性。倾向值匹配算法便是通过控制倾向值来达到控制和匹配的目的。在基于倾向值的匹配过程中混淆变量被控制起来了,那两组微博影响力的差异就只能归因于是否采用了策略,进而能评估策略的有效性。具体的,算法流程如表4-3。


步骤一:计算倾向值。我们将混淆变量纳入逻辑斯地回归模型来计算倾向值,倾向值指的是计算研究个体在控制可观测到的混淆变量的情况下\cite{"倾向值匹配与因果推论"},受到某种自变量影响的条件概率。本文中指的是这条微博在其特征固定的情况下,采用某项策略的条件概率:
\begin{equation}e_i = P(T_i = 1 | X_i)\end{equation}

步骤二:基于倾向值进行匹配。根据是否采用了某项策略,微博整体可分为两个样本集合。对两个样本集合中的微博根据倾向值采用匹配算法进行匹配。匹配算法有多种,从匹配数量角度有一对一匹配、多对一匹配和多对多匹配,从匹配方法角度有贪心匹配、最近邻匹配、基于卡尺的最近邻匹配等等\cite{"an introduction to propensity"}。本文中采用一对多匹配和基于卡尺的最近邻匹配。在基于卡尺的最近邻匹配中,卡尺距离通常为倾向值对数的标准差的0.2倍,这个距离被证明在多种参数设置中会产生最优的风险差异估计。通过计算,在本章的实验中,卡尺距离为0.09。

步骤三:评估策略效果并检验有效性。在基于倾向值对两组微博进行配对后,通过计算所有配对的微博的影响力差异,便能得到策略带来的有效水平。在我们的数据中,因为我们关注的是最终采用了某项策略的微博,因此我们用平均干预效果ATT(Average Treatment effect for the Treated)来衡量策略效果。然后用t检验(t-test)来检验干预效果的显著性水平,即检验两组微博中的微博影响力水平是否有显著差异。
\begin{equation}ATT = E[Y(1) - Y(0) | T = 1]\end{equation}

步骤四:平衡性诊断。倾向值匹配的目的是通过控制倾向值近似来达到混淆变量类似的目的,因此检验配对微博在混淆变量上是否有显著差异是必不可少的,它代表配对的微博在混淆变量上的相似程度。平衡性检验能评估倾向值模型是否充分恰当的进行了应用。标准差用于量化采用和没采用策略的两组微博混淆变量平均值的差异\cite {14},其对连续变量和二值变量的公式定义分别如式4-1和4-2。最后,累积密度图和分位数位图(qq图)用于比较采用策略和没采用策略两组混淆变量的分布情况。

\begin{equation}d = \frac{(\overline{x}_{treated} - \overline{x}_{untreated})}{\sqrt{\frac{s_{treated}^2 + s_{untreated}^2}{2}}}\end{equation}
其中,$\overline{x}_{treated}$和$\overline{x}_{untreated}$分别对应采用策略和没采用策略组样本的连续混淆变量的均值, $s_{treated}^2$和$s_{untreated}^2$分别对应采用策略和没采用策略组样本的连续混淆变量的方差.
\begin{equation}d = \frac{(\hat{p}_{treated} - \hat{p}_{untreated})}{\sqrt{\frac{\hat{p}_{treated}(1 - \hat{p}_{treated}) + \hat{p}_{untreated}(1 - \hat{p}_{untreated})}{2}}}\end{equation}
其中,$\hat{p}_{treated}$和$\hat{p}_{untreated}$分别对应采用策略和没采用策略组样本的二值混淆变量的均值。


\section{结果分析}
通过倾向值匹配算法,得到针对不同推广策略的干预效果,可以更好的指导演员进行推广,提高电视剧微博热度。

(1) 推广周期。按推广周期计算筹备阶段、首播阶段和首播后阶段的平均干预效果ATT如表5所示。可以看到,在筹备阶段发布的微博的平均干预效果为负值,且t检验结果显著,说明筹备阶段发布的推广微博影响力不如非筹备阶段,即首播阶段和首播后阶段。而首播阶段发布微博的影响力效果虽然为负值,但是经过t检验,与非首播阶段发布微博的影响力效果没有显著差异。与二者相比,首播后阶段发布微博的影响力效果与筹备阶段和首播阶段相比有显著差异,会显著提高。分析原因可能是粉丝们在筹备阶段和首播阶段只能看到关于电视剧的少量信息和花絮,参与感较低,导致关注度不高;而当用户看过电视剧后,在演员利用微博进行电视剧推广时,粉丝们可以针对剧情、人物等积极参与讨论,对微博内容也可以有更多的态度和观点,此时进行推广,效果会更好。

\begin{table}[H]
\centering
\caption{Comparison of each promotion period}
\begin{tabular}{|c|c|c|c|} \hline
Promotion Period& ATT & Significance\\ \hline
preparation period & -674.204& 0.000\\ \hline
premiere period & -182.854& 0.260\\ \hline
post-premiere period & 757.065 & 0.000\\ 
\hline\end{tabular}

\end{table}

(2) 推广时间。计算各个推广时间干预效果ATT和t检验的结果如表6所示。与想象中不同的是,早晨发布推广微博获得的推广效果要显著好于中午和晚上发布。原因可能是,推广信息具有时效性,对当天来说,早晨发布的微博一天中被看到的时长和概率是最大的,到了第二天,消息的时效性大大降低,导致粉丝的参与度大大降低。因此建议演员更多的在早晨发布推广微博。

\begin{table}[H]
\centering
\caption{Comparison of each promotion time}
\begin{tabular}{|c|c|c|c|} \hline
Promotion Time& ATT & Significance\\ \hline
morning & 879.083& 0.000\\ \hline
afternoon & -138.270& 0.036\\ \hline
night & -314.376 & 0.000\\ 
\hline\end{tabular}

\end{table}

(3) 互动模式。
按互动模式将演员推广行为分为与其他主演互动,与官微互动,原创非互动和其他互动4种方式。经过倾向值匹配算法,得到结果如表7所示。可以 看到演员与其他主演互动、与官微互动时,平均干预效果都有所增加,都对推广有显著效果。因为主演和主演互动本身就在制造话题性,与官微互动,会转发一些电视剧情节、花絮及与演员相关的信息,信息量较大,会增加粉丝参与。而原创非互动模式与非原创相比,平均干预效果显著提高,推广效果提高的最大。分析演员的原始微博可知,原创微博更能表达演员的情感和对粉丝参与的情感呼吁,因而更能得到粉丝的支持和参与,推广效果也会更好。与之对比,转发其他视频网站、粉丝的微博等其他互动方式的推广效果明显不如另外3种互动方式,不管从携带信息量还是从演员情感角度 看,其他互动方式的宣传效果都会较差。


\begin{table}[H]
\centering
\caption{Comparison of each interaction mode}
\begin{tabular}{|c|c|c|c|} \hline
Interaction Mode& ATT & Significance\\ \hline
actors & 759.315& 0.000\\ \hline
official accounts & 3413.222& 0.000\\ \hline
original & 12007.880& 0.000\\ \hline
other & -7195.497 & 0.000\\ 
\hline\end{tabular}

\end{table}

综上可知,在演员进行推广时,建议从宣传周期上,选择首播后阶段发布更多的微博进行推广;从发布 时间上,多选择上午发布,增加用户参与度;在微博互动方式上,在时间允许的情况下,多发原创微博,多与官微和演员互动,能更好的推广电视剧, 提高微博话题热度,增加电视剧点击量。

\section{模型显著性及平衡性检验}
平衡性检验用来评估倾向值匹配模型是否恰当的进行了应用。对每项策略计算标准差。结果显示标准差的取值范围为0到0.049,意味着在采用策略组和没采用策略组,配对的连续变量和非连续变量的平均值非常类似。总之,上述分析显示倾向值匹配算法在我们的数据中被恰当的进行了应用。因此在我们研究的策略和利用的混淆变量的基础上,对观测数据得到的结论是在电视剧首播后期、早晨发布原创或与主演和官微互动的微博,能达到更好的宣传效果,提高粉丝的参与度和话题热度。








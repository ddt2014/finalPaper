\chapter{总结与展望}

\section{工作总结}

随着社交媒体的不断发展,在微博、微信等社交媒体上进行宣传,已经成为一种重要的推广手段。尤其在电影宣传领域,越来越多的宣传方通过社交媒体推送广告,极大地促进了电影票房的增长。但是对于电视剧来说,由于其播放周期长、重复播放等特点,虽然同属于影视剧领域,但是与电影的宣传策略和推广特点存在不同之处。因此,本文选取演员在微博上对电视剧的推广作为研究对象,比较不同的推广策略对最终推广效果的影响。

在微博上,主要是依靠电视剧官方账号和电视剧的主要演员创建与电视剧相关的话题,发布相关微博来进行推广,不断推动电视剧话题热度的发展。但是对于不同的电视剧和不同的演员来说,不同的推广策略能够获得的推广效果是不同的。同样的推广策略应用到不同的电视剧或者演员身上也会获得不同的效果。因此,在分析各个推广策略所获得的推广效果时,需要排除与电视剧和演员相关属性对推广结果的干扰,才能分析推广策略和推广效果之间因果关系。

因此,本文利用倾向值匹配算法来分析不同推广策略与推广效果之间的因果关系,将电视剧和演员信息作为混淆变量纳入模型当中,通过计算各个微博的倾向值,并根据倾向值进行匹配,得到匹配的微博对。因为匹配过程中倾向值几乎相等,从而认为混淆变量相似,那么每对微博之间的唯一差异在于是否采取了该项策略,而与其他因素无关。通过比较这些配对微博的影响力,就可以判断推广策略对推广效果所造成的影响。

同时,本文还根据电视剧话题在微博上的演化规律,提出了基于话题演化的改进模型,将所有微博以电视剧的首播日期为分界点,划分为平稳期和爆发期两组,对两组内的微博分别利用倾向值匹配模型,识别有效推广策略。另外在改进模型中,还引入了更多的电视剧和演员的基本信息作为混淆变量,使得分析结果更准确。

本文利用多种方法从爱奇艺、豆瓣和微博上爬取电视剧和演员相关的数据。首先对数据进行了整理和清洗,然后进行了策略和挖掘,比如,提出微博影响力指数衡量一条微博带来的影响力;分析微博影响力与话题热度之间的相关性;发现演员宣传电视剧的推广模式和推广策略,并发现了一些推广规律等等。之后,利用基于话题演化模型的倾向值匹配算法,分析推广策略与推广效果之间的因果关系。通过分析可以看出,在平稳期内,演员在晚上发布原创微博和在爆发期内,演员在下午发布原创微博能够获得更好的推广效果。最后,通过对模型进行显著性及平衡性检验,验证了该算法在数据集上实际应用的可行性和准确性。

\section{未来展望}

在现已完成工作的基础上,下一步有如下方向可以进行改进:

(1)除了本文选取的混淆变量,还有其他因素会影响推广效果,比如导演、电视剧类型、发布微博的文本内容和情感倾向等等。下一步工作中,我们可以将更多可能影响推广效果的因素加入到混淆变量,使得因果分析结果更准确。同时,为了判断在分析过程当中是否忽略了某些信息,未能将其他与推广效果相关的信息纳入到混淆变量中,还可以对结果进行敏感性分析。

(2)在对电视剧的推广过程中,除了演员和官方微博,还有一些大V用户也发挥了信息专家的作用,不断推动电视剧话题热度提高,比如电视剧导演、播出平台官方微博、网站官方微博等等,下一步可以获取这些用户对电视剧的推广行为数据,分析他们在社交媒体上所发挥的作用。

(3)利用该算法还可以将收集到的数据整合成为一个建议系统,通过输入待推广的电视剧名称和要发布微博进行推广的演员名称,就可以根据电视剧和演员的自身特点,个性化地分析适合其的推广策略,帮助其选取推广效果最好的策略发布微博,使该方法具有一定的实用意义。





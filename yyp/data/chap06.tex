\chapter{总结与展望}

\section{工作总结}

随着社交网络的不断发展,在微博、微信等社交媒体上进行宣传,已经成为一种重要的推广手段。尤其在电影宣传领域,越来越多的宣传方通过社交媒体推送广告,极大地促进了电影票房的增长。但是对于电视来说,由于其播放周期长、重复播放等特点,虽然同属于影视剧领域,但是与电影的宣传策存在一定的不同之处。因此,本文选取在微博上对电视剧的推广作为研究对象,比较不同的推广策略对最终推广效果的影响。

在微博上,主要是依靠电视剧官方账号和电视剧的主要演员创建与电视剧相关的话题,发布相关微博来进行推广。但是对于不同的电视剧和不同的演员来说,不同的推广策略能够获得的推广效果是不同的。同样的推广策略应用到不同的电视剧或者演员身上也会获得不同的效果。因此,在分析各个推广策略所获得的推广效果时,需要排除与电视剧和演员相关其他属性对结果的影响,仅仅分析策略和效果之间的直接关系。

因此,本文利用倾向值匹配算法来分析不同推广策略与推广效果之间的因果关系,将电视剧和演员信息作为混淆变量纳入模型当中,通过计算各个微博的倾向值,并对相同的倾向值进行匹配,从而可以找出成对的微博,每对微博之间的唯一差异在于是否采取了该项策略,而与其他因素无关。通过比较这对推广微博的影响力,就可以判断是否采取策略对推广效果所造成的影响。

同时,本文还根据电视剧话题在微博上的演化规律,提出了基于话题演化的改进模型,将所有微博以电视剧的首播日期为分界点,划分为平稳期和爆发期两组,对各组内的所有微博进行倾向值匹配,比较不同策略的推广效果。另外在改进模型中,还引入了更多的电视剧和演员的基本信息作为混淆变量,使得分析结果具有更强的因果关系。

本文通过网络爬虫,提取了2016年1月1日至2016年12月31日之间上映的313部电视剧和这些电视剧的1121位主演姓名,利用基于话题演化模型的倾向值匹配算法,分析推广时间以及互动模式两种推广模式与推广效果之间的因果关系。通过分析可以看出,在平稳期内,演员在晚上发布原创微博和在爆发期内,演员在下午发布原创微博能够获得更好的推广效果。并且通过进行模型显著性及平衡性检验,验证了该算法在数据集上实际应用的可行性和准确性。

\section{未来展望}

在后续工作中,为了验证算法的准确性,判断在分析过程当中是否忽略了某些信息,未能将其他与推广效果相关的信息纳入到混淆变量中进行讨论,还可以对结果进行敏感性分析,来判断各个推广策略对结果的影响程度。同时目前提取的关于电视剧和演员的社交网络数据还不算太多,在下一步的工作中,还可以进一步扩大数据获取来源,提取更多的电视剧以及演员信息,收集更多的微博数据,在更大的数据集下验证算法的准确性。同时,利用该算法还可以将收集到的数据整合成为一个建议系统,通过数据待推广的电视剧名称和要发布微博进行推广的演员名称,即可以根据电视剧和演员的自身特点,个性化地分析适合其的推广策略,帮助其选取推广效果最好的策略发布微博,使该方法具有一定的实用意义。
\begin{resume}

  \resumeitem{个人简历}

  1989 年 8 月 12 日出生于辽宁省阜新市。

  2008 年 9 月考入清华大学自动化系,2012年 7 月本科毕业并获得工学学士学位。

  2014 年 9 月免试进入清华大学计算机科学与技术系攻读硕士学位至今。

  \researchitem{发表的学术论文} % 发表的和录用的合在一起
% 学位论文写作指南:
% 在学期间发表的学术论文分以下三部分按顺序分别列出,每部分之间空 1
% 行,序号可连续排列
% 1. 已经刊载的学术论文(本人是第一作者,或者导师为第一作者本人是第二作者)
% 2. 尚未刊载,但已经接到正式录用函的学术论文(本人为第一作者,或者
% 导师为第一作者本人是第二作者)。
% 3. 其他学术论文。可列出除上述两种情况以外的其他学术论文,但必须是
% 已经刊载或者收到正式录用函的论文。
  \begin{publications}
  \item Ding Xu, Tao Pin. Synchronization Detection of Multicamera System based on LED Matrix. International Conference on Embedded Software and Systems, 2016
  \item 丁旭, 陶品. 基于LED点阵的摄像机拍摄时间检测方法. Conference on Pervasive Computing of China, 2016
  \end{publications}

\end{resume}

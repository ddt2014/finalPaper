\begin{resume}

  \resumeitem{个人简历}

  1991 年 1 月 6 日出生于山东省栖霞市。

  2010 年 9 月考入北京邮电大学计算机系,2014年 7 月本科毕业并获得工学学士学位。

  2014 年 9 月免试进入清华大学计算机科学与技术系攻读硕士学位至今。

  \researchitem{发表的学术论文} % 发表的和录用的合在一起
% 学位论文写作指南:
% 在学期间发表的学术论文分以下三部分按顺序分别列出,每部分之间空 1
% 行,序号可连续排列
% 1. 已经刊载的学术论文(本人是第一作者,或者导师为第一作者本人是第二作者)
% 2. 尚未刊载,但已经接到正式录用函的学术论文(本人为第一作者,或者
% 导师为第一作者本人是第二作者)。
% 3. 其他学术论文。可列出除上述两种情况以外的其他学术论文,但必须是
% 已经刊载或者收到正式录用函的论文。
  \begin{publications}
  \item 阎一萍, 姚鑫,孙立峰. 电视剧演员的社交推广行为的影响力模型. 第二十五届全国多媒体技术学术会议, 2016
  \item Yiping Yan, Lifeng Sun, and Xin Yao. Evaluating Actors' Promotion Behaviors for TV Series on Social Networks. Proceedings of the International Conference on Internet Multimedia Computing and Service. ACM, 2016.
  \item Yiping Yan, Lifeng Sun, and Xin Yao. How to Promote TV Series? Evaluating Actors’ Behavior on Social Media. The Third IEEE International Conference on Multimedia Big Data, Best Student Paper, 2017
  \end{publications}

\end{resume}

\ctitle{多摄像头系统的时间同步}
% 根据自己的情况选,不用这样复杂
\makeatletter
\ifthu@bachelor\relax\else
  \ifthu@doctor
    \cdegree{工学博士}
  \else
    \ifthu@master
      \cdegree{工学硕士}
    \fi
  \fi
\fi
\makeatother


\cdepartment[计算机]{计算机科学与技术系}
\cmajor{计算机科学与技术}
\cauthor{阎一萍}
\csupervisor{孙立峰教授}
\etitle{the Synchronization of Multicamera System}
\edegree{Master of Engineering}
\emajor{Computer Science and Technology}
\eauthor{Yan Yiping}
\esupervisor{Professor Sun LiFeng}
% 这个日期也会自动生成,你要改么?
% \edate{December, 2005}

% 定义中英文摘要和关键字
\begin{cabstract}


\end{cabstract}

\ckeywords{阵}

\begin{eabstract}

With the development of camera technology, in order to achieve higher performance and more functions, multi-camera system has become an important research direction. There are more and more researches on the multi-camera system. In order to meet the needs of the system application, the system must achieve the shooting time synchronization, to accurately control cameras to shoot at a certain moment.

In order to realize the time synchronization of multi-camera system, this paper proposes a high precision method to detect the synchronization accuracy by using a FPGA based LED matrix as a detector. The method is divided into two parts: camera shooting time detection and multi-camera system time synchronization control.

This paper use a detection system composed by FPGA and LED dot matrix to detect the camera shooting time. The LED lights in the LED matrix continue to change forming a state sequence. Depending on the encoding method, it is possible to determine the position of each matrix state in the sequence. When shooting the LED dot matrix using the camera, the position of the state in the state sequence can be determined, and the shooting time of the camera can be detected. For different encoding methods, different detection accuracy can be obtained, the camera also has different performance and parameter requirements. In this paper, five encoding methods are designed, and the results are compared according to experiments. With this method, the detection system needs no data communication with the camera, so the camera does not have to equip special hardware interface. This method can be applied to most of the camera. At the same time, the method can be applied during the normal working process of the camera, without pausing or adjusting the camera.

After the camera's shooting time is detected, the multi-camera system can be adjusted synchronously. First of all, the server identifies images of the LED dot matrices captured by cameras. So the server can get cameras' shooting time and calculate the time intervals between the various cameras. According the reference camera, the server then can change the cameras' shooting time by pausing cameras or adjusting the frame rate. So that the cameras in the system can shoot at the same time. And this method can continuously improve the synchronization accuracy by multiple iterative verification. This synchronization method can achieve a high synchronization accuracy for the use of the shooting time detection method. At the same time, the cameras' computing performance requirements are lower and the method flexibility is higher.

This paper also builds a scalable multi-camera system, using the Raspberry to to control the camera and the image processing server to control the system. With this system, the above detection and synchronization method has been verified.

\end{eabstract}

\ekeywords{multicamera system, synchronization, shooting time detection, LED matrix}






































